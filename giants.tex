\subsection{Caratteristiche pianeti giganti}

\begin{frame}{Pianeti giganti: caratteristiche orbitali}\tolbf
\begin{table}[!ht]
\pgfplotstabletypeset[skip rows between index={0}{4},
every head row/.style={
 %before row={},
 %every last row/.style={after row=\bottomrule},
 after row={\midrule}
},
every 2 row/.style={after row=\midrule},
every last row/.style={after row=\bottomrule},
every first column/.style={column type/.add={|}{}},
every last column/.style={column type/.add={}{|}},
%columns/0/.style = {column type/.add={|}{}},
%columns/5/.style = {column type/.add={|}{}},
%columns/0/.style={string type},
display columns/1/.style={column name={$r_{p\odot}(\SI{e8}{\kilo\meter})$},clear infinite},
display columns/2/.style={column name={e},clear infinite},
display columns/3/.style={column name={i},clear infinite},
display columns/4/.style={column name={$\tau_{sid}$},clear infinite},
display columns/5/.style={column name={obliquity},clear infinite},
display columns/planets/.style={column name={pianeta},string type},
create on use/planets/.style={create col/set list={Mercury, Venus, Earth, Mars,Jupiter,Saturn,Uranus, Neptune}},
columns/planets/.style={string type},
columns={planets, dps, e, i, Ps, obl},
/pgf/number format/precision=3
     ]{solarsystemmain-orbit.txt} %%%
%\captionof{table}{Caratteristiche pianeti giganti.}\label{tab:terrestrial planets}
\end{table}
\end{frame}

\subsection{Jupiter system.}

\begin{frame}{Resonances (da distribuire)}


The resonance ensures that, when they approach perihelion and Neptune's orbit, Neptune is consistently distant (averaging a quarter of its orbit away). Other (much more numerous) Neptune-crossing bodies that were not in resonance were ejected from that region by strong perturbations due to Neptune. There are also smaller but significant groups of resonant trans-Neptunian objects occupying the 1:1 (Neptune trojans), 3:5, 4:7, 1:2 (twotinos) and 2:5 resonances, among others, with respect to Neptune.

The Titan Ringlet within Saturn's C Ring represents another type of resonance in which the rate of apsidal precession of one orbit exactly matches the speed of revolution of another. The outer end of this eccentric ringlet always points towards Saturn's major moon Titan.

A Kozai resonance occurs when the inclination and eccentricity of a perturbed orbit oscillate synchronously (increasing eccentricity while decreasing inclination and vice versa). This resonance applies only to bodies on highly inclined orbits; as a consequence, such orbits tend to be unstable, since the growing eccentricity would result in small pericenters, typically leading to a collision or (for large moons) destruction by tidal forces.

A Laplace resonance is a three-body resonance with a 1:2:4 orbital period ratio (equivalent to a $4:2:1$ ratio of orbits). The term arose because Pierre-Simon Laplace discovered that such a resonance governed the motions of Jupiter's moons Io, Europa, and Ganymede. It is now also often applied to other 3-body resonances with the same ratios, such as that between the extrasolar planets Gliese 876 c, b, and e. Three-body resonances involving other simple integer ratios have been termed "Laplace-like" or "Laplace-type".

A Lindblad resonance drives spiral density waves both in galaxies (where stars are subject to forcing by the spiral arms themselves) and in Saturn's rings (where ring particles are subject to forcing by Saturn's moons).

Several prominent examples of secular resonance involve Saturn. A resonance between the precession of Saturn's rotational axis and that of Neptune's orbital axis (both of which have periods of about 1.87 million years) has been identified as the likely source of Saturn's large axial tilt ($26.7\deg$). Initially, Saturn probably had a tilt closer to that of Jupiter ($3.1\deg$). The gradual depletion of the Kuiper belt would have decreased the precession rate of Neptune's orbit; eventually, the frequencies matched, and Saturn's axial precession was captured into the spin-orbit resonance, leading to an increase in Saturn's obliquity. (The angular momentum of Neptune's orbit is 104 times that of Saturn's spin, and thus dominates the interaction.)

Examples are the $1:2:4$ resonance of Jupiter's moons Ganymede, Europa and Io, and the $2:3$ resonance between Pluto and Neptune. Unstable resonances with Saturn's inner moons give rise to gaps in the rings of Saturn. The special case of 1:1 resonance (between bodies with similar orbital radii) causes large Solar System bodies to eject most other bodies sharing their orbits; this is part of the much more extensive process of clearing the neighbourhood, an effect that is used in the current definition of a planet.

The Titan Ringlet within Saturn's C Ring represents another type of resonance in which the rate of apsidal precession of one orbit exactly matches the speed of revolution of another. The outer end of this eccentric ringlet always points towards Saturn's major moon Titan.

A Kozai resonance occurs when the inclination and eccentricity of a perturbed orbit oscillate synchronously (increasing eccentricity while decreasing inclination and vice versa). This resonance applies only to bodies on highly inclined orbits; as a consequence, such orbits tend to be unstable, since the growing eccentricity would result in small pericenters, typically leading to a collision or (for large moons) destruction by tidal forces.

In an example of another type of resonance involving orbital eccentricity, the eccentricities of Ganymede and Callisto vary with a common period of 181 years, although with opposite phases.

\end{frame}


\begin{frame}{Giove}


\end{frame}

\begin{wordonframe}{giove}
Il diametro di Giove $D_G\approx\SI{e5}{\kilo\meter}$ visto dalla terra sottende angolo $\alpha\approx\SI{1.5e-4}{\radian}=\ang{;;30}$.

Caratteristiche dell'orbita:
\begin{itemize}
    \item $\Pi_{360}\approx\SI{12}{\year}$.
    \item $d_{G\odot}\approx\SI{5}{\astronomicalunit}$.
    \item $e=0.048$ ($d(P,F)=ed(P,r), b^2=(1-e^2)a^2$)
\end{itemize}
\end{wordonframe}



\subsection{Saturn system.}

\subsection{Uranus}


\subsection{Neptune}

