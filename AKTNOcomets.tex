\section{Corpi minori del sistema solare interno}

\begin{frame}{Scenario formazione: corpi minori}
\begin{itemize}
\item Why there is so little mass remaining in asteroid region ($\approx\SI{3}{\astronomicalunit}$)? 
\item Why is this mass spread over so many body??
\item Why most of asteroid orbits are more inclinated/eccentric than that of major planets?
\item Why are asteroid so compoitionally diverse?
\item Kuiper Belt require that planetesimal exist beyond Neptune; Why the abrupt cut-oof beyond Neptune?
\item Oort cloud: Mass of solid material eject from planetary (giants) region \numrange{1}{1000}$\mearth{}$
\end{itemize}

\end{frame}

\begin{wordonframe}{Asteroidi, comete, TNO, ogetti della fascia di Kuiper-Edgeworth.}

\end{wordonframe}

\begin{frame}{Resonances}
Examples are the $1:2:4$ resonance of Jupiter's moons Ganymede, Europa and Io, and the $2:3$ resonance between Pluto and Neptune. Unstable resonances with Saturn's inner moons give rise to gaps in the rings of Saturn. The special case of 1:1 resonance (between bodies with similar orbital radii) causes large Solar System bodies to eject most other bodies sharing their orbits; this is part of the much more extensive process of clearing the neighbourhood, an effect that is used in the current definition of a planet.

The orbits of Pluto and the plutinos are stable, despite crossing that of the much larger Neptune, because they are in a 2:3 resonance with it.

In the asteroid belt beyond 3.5 AU from the Sun, the 3:2, 4:3 and 1:1 resonances with Jupiter are populated by clumps of asteroids (the Hilda family, the few Thule asteroids, and the extremely numerous Trojan asteroids, respectively).

In the asteroid belt within 3.5 AU from the Sun, the major mean-motion resonances with Jupiter are locations of gaps in the asteroid distribution, the Kirkwood gaps (most notably at the $3:1$, $5:2$, $7:3$ and $2:1$ resonances).

In the rings of Saturn, the Cassini Division is a gap between the inner B Ring and the outer A Ring that has been cleared by a $2:1$ resonance with the moon Mimas. (More specifically, the site of the resonance is the Huygens Gap, which bounds the outer edge of the B Ring.)
    In the rings of Saturn, the Encke and Keeler gaps within the A Ring are cleared by 1:1 resonances with the embedded moonlets Pan and Daphnis, respectively. The A Ring's outer edge is maintained by a destabilizing $7:6$ resonance with the moon Janus.

A Laplace resonance is a three-body resonance with a 1:2:4 orbital period ratio (equivalent to a $4:2:1$ ratio of orbits). The term arose because Pierre-Simon Laplace discovered that such a resonance governed the motions of Jupiter's moons Io, Europa, and Ganymede. It is now also often applied to other 3-body resonances with the same ratios, such as that between the extrasolar planets Gliese 876 c, b, and e. Three-body resonances involving other simple integer ratios have been termed "Laplace-like" or "Laplace-type".

A Lindblad resonance drives spiral density waves both in galaxies (where stars are subject to forcing by the spiral arms themselves) and in Saturn's rings (where ring particles are subject to forcing by Saturn's moons).

Several prominent examples of secular resonance involve Saturn. A resonance between the precession of Saturn's rotational axis and that of Neptune's orbital axis (both of which have periods of about 1.87 million years) has been identified as the likely source of Saturn's large axial tilt ($26.7\deg$). Initially, Saturn probably had a tilt closer to that of Jupiter ($3.1\deg$). The gradual depletion of the Kuiper belt would have decreased the precession rate of Neptune's orbit; eventually, the frequencies matched, and Saturn's axial precession was captured into the spin-orbit resonance, leading to an increase in Saturn's obliquity. (The angular momentum of Neptune's orbit is 104 times that of Saturn's spin, and thus dominates the interaction.)

The perihelion secular resonance between asteroids and Saturn  helps shape the asteroid belt. Asteroids which approach it have their eccentricity slowly increased until they become Mars-crossers, at which point they are usually ejected from the asteroid belt by a close pass to Mars. This resonance forms the inner and "side" boundaries of the asteroid belt around 2 AU, and at inclinations of about $20\deg$.

The Titan Ringlet within Saturn's C Ring represents another type of resonance in which the rate of apsidal precession of one orbit exactly matches the speed of revolution of another. The outer end of this eccentric ringlet always points towards Saturn's major moon Titan.

A Kozai resonance occurs when the inclination and eccentricity of a perturbed orbit oscillate synchronously (increasing eccentricity while decreasing inclination and vice versa). This resonance applies only to bodies on highly inclined orbits; as a consequence, such orbits tend to be unstable, since the growing eccentricity would result in small pericenters, typically leading to a collision or (for large moons) destruction by tidal forces.

In an example of another type of resonance involving orbital eccentricity, the eccentricities of Ganymede and Callisto vary with a common period of 181 years, although with opposite phases.1

\end{frame}

\begin{wordonframe}[allowframebreaks]{resonances}

In celestial mechanics, an orbital resonance occurs when two orbiting bodies exert a regular, periodic gravitational influence on each other, usually due to their orbital periods being related by a ratio of two small integers. The physics principle behind orbital resonance is similar in concept to pushing a child on a swing, where the orbit and the swing both have a natural frequency, and the other body doing the "pushing" will act in periodic repetition to have a cumulative effect on the motion. Orbital resonances greatly enhance the mutual gravitational influence of the bodies, i.e., their ability to alter or constrain each other's orbits. In most cases, this results in an unstable interaction, in which the bodies exchange momentum and shift orbits until the resonance no longer exists. Under some circumstances, a resonant system can be stable and self-correcting, so that the bodies remain in resonance.

A mean-motion orbital resonance occurs when two bodies have periods of revolution that are a simple integer ratio of each other. Depending on the details, this can either stabilize or destabilize the orbit. Stabilization may occur when the two bodies move in such a synchronised fashion that they never closely approach. For instance:

The resonance ensures that, when they approach perihelion and Neptune's orbit, Neptune is consistently distant (averaging a quarter of its orbit away). Other (much more numerous) Neptune-crossing bodies that were not in resonance were ejected from that region by strong perturbations due to Neptune. There are also smaller but significant groups of resonant trans-Neptunian objects occupying the 1:1 (Neptune trojans), 3:5, 4:7, 1:2 (twotinos) and 2:5 resonances, among others, with respect to Neptune.

Orbital resonances can also destabilize one of the orbits. For small bodies, destabilization is actually far more likely.

In the asteroid belt within 3.5 AU from the Sun, the major mean-motion resonances with Jupiter are locations of gaps in the asteroid distribution, the Kirkwood gaps (most notably at the $3:1$, $5:2$, $7:3$ and $2:1$ resonances). Asteroids have been ejected from these almost empty lanes by repeated perturbations. However, there are still populations of asteroids temporarily present in or near these resonances. For example, asteroids of the Alinda family are in or close to the $3:1$ resonance, with their orbital eccentricity steadily increased by interactions with Jupiter until they eventually have a close encounter with an inner planet that ejects them from the resonance.

Most bodies that are in resonance orbit in the same direction; however, a few retrograde damocloids have been found that are temporarily captured in mean-motion resonance with Jupiter or Saturn. Such orbital interactions are weaker than the corresponding interactions between bodies orbiting in the same direction.

A secular resonance occurs when the precession of two orbits is synchronised (usually a precession of the perihelion or ascending node). A small body in secular resonance with a much larger one (e.g. a planet) will precess at the same rate as the large body. Over long times (a million years, or so) a secular resonance will change the eccentricity and inclination of the small body.

Numerical simulations have suggested that the eventual formation of a perihelion secular resonance between Mercury and Jupiter has the potential to greatly increase Mercury's eccentricity and possibly destabilize the inner Solar System several billion years from now.

The Titan Ringlet within Saturn's C Ring represents another type of resonance in which the rate of apsidal precession of one orbit exactly matches the speed of revolution of another. The outer end of this eccentric ringlet always points towards Saturn's major moon Titan.

A Kozai resonance occurs when the inclination and eccentricity of a perturbed orbit oscillate synchronously (increasing eccentricity while decreasing inclination and vice versa). This resonance applies only to bodies on highly inclined orbits; as a consequence, such orbits tend to be unstable, since the growing eccentricity would result in small pericenters, typically leading to a collision or (for large moons) destruction by tidal forces.
\end{wordonframe}


\subsection{Main Belt}

\begin{frame}{Caratteristiche main belt}
Sono la principale sorgente di meteoriti.
Classificazione in base a indici di colore (fotometria multibanda):
\begin{itemize}
    \item C: Condriti carbonacee.
    \item S: asteroidi con bassa albedo
\end{itemize}
Orbite comprese tra Marte e Giove. Il primo \'e 1Ceres, diametro circa \SI{1000}{\kilo\meter}.
Classificazione dinamica:
\begin{itemize}
\item Forniscono informazioni sui processi di formazione del sistema solare.
\item Fascia principale: tra Marte e Giove.
\item Rapido spopolamento man mano che ci si avvicina a Giove.
\item Lacune di Kirkwood: per valori del semi-asse maggiore risonanti con quello di Giove
(Risonanza: rapporto periodo orbitale razionale non troppo distante da 1.)
\end{itemize}
Da Terra si vedono come sorgenti puntiformi. Forma, dimensioni e caratteristiche superficiali: tecniche interferometriche e fenomeni di occultamento.

La spettro di riflessione \'e vario: dipende dalle caratteristiche chimiche e fisiche della superficie.
\end{frame}

\begin{wordonframe}{orbita, spin, albedo, etc}

\end{wordonframe}

\begin{frame}{famiglie collisionali}

\end{frame}

\begin{wordonframe}{evoluzione collisionale}

\end{wordonframe}

\subsection{NEO/Meteors}

\begin{frame}{NEO}
Orbita pi\'u interna, incrocia anche l'orbita della Terra.
\end{frame}

\begin{frame}{Meteoriti}

\end{frame}

\begin{wordonframe}{spettro di massa dei meteoriti}

\end{wordonframe}

\subsection{Kuiper, TNO, comets}

\begin{frame}{Reservoir di corpi}

\end{frame}

\begin{wordonframe}{Jupiter class comets}

\end{wordonframe}


\subsubsection{Troiani.}

Sono oggetti che hanno lo stesso semi-asse di Giove ma spostati di \ang{+-60} nell'orbita cio\'e nei punti Lagrangiani.

\subsubsection{Comete.}
Orbite eccentriche.

Il cambiamento delle loro propriet\'a dipende dalla distanza dal Sole: ricche di sostanze volatili

Originaria di una fascia esterna di corpi minori compresa tra asteroidi e TNO, in seguito a incontri ravvicinati con corpi maggiori si sono spostate in orbite che raggiungono all'afelio i confini del sistema solare (Nube di Oort approx \SI{e5}{\astronomicalunit}: quando diventa prevalente l'attrazione delle stelle vicine).

\subsubsection{Centauri.}

Centauri: orbite comprese tra Giove e Nettuno. La zona \'e dinamicamente instabile e porta in orbite cometaria.

\subsubsection{Trans-Neptunian object: Fascia di Kuiper-Edgeworth.}

Oltre Nettuno di hanno i TNO.

Un sottogruppo dei TNO, i plutini, sono in risonanza $3:2$ con Nettuno (come Plutone).

\subsection{Search motivation}

\begin{itemize}
\item Why acretional formation of solar system planet objects should stop at Neptuno's distance.
\item The jupiter family comets are almost on planar orbits with low inclination on ecliptic plane (plane of solar system). This is inesplicable if the source is far away and isotropic, so we may may postulate a a disc of cometary object beyond Neptun.
\end{itemize}

\subsection{Comets}

\subsection{Asteroids}

\subsection{Short summary}
\begin{itemize}
\item Originate from primitive bodies: TNO, comets and asteroids.
\item Span 20 order of magnitude in mass
\item Observed ground or space based optically or IR: reflected solar light or thermal emission by small interplanetary dust arranged in a disk like structure on the ecliptic plane.
Zodiacal light is caused by reflected light at large angle, false corona light is attributed to small angle diffractive scattering.
\end{itemize}


