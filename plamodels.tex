\section{Classificazione.}

\begin{frame}{Pianeti sistema solare}
\end{frame}

\begin{wordonframe}{classificazione: pianeti sistema solare}
Un pianeta ha le seguenti caratteristiche
\begin{itemize}
    \item \'E in orbita attorno ad una stella di riferimento.
    \item \'E abbastanza massiccio da essere dominato dalle forze di gravit\'a: forma di equilibrio ''idrostatica''.
    \item Ha completamente ripulito la regione del sistema intorno alla sua orbita. Altrimenti \'e un pianeta nano.
\end{itemize}

\end{wordonframe}

\section{Modello di pianeta}

\subsection{Energia assorbita. Temperatura efficace/Radiative transfer}

Suppongo una distribuzione uniforme sulla superficie del pianeta.

La temperatura efficace del pianeta nell'approssimazione di corpo nero
\begin{align*}
&4\pi R_P^2\sigma T_{eff,P}^4=(1-A)\frac{L^*R_P^2}{4r_{P*}^2}\\
&=(1-A)\frac{\pi R_*^2T_{eff,*}^4}{r_{P*}^2}R_P^2\\
&T_{eff,P}=(\frac{1-A}{A})\expy{\frac{1}{4}}(\frac{R_*}{r_{P*}})\expy{\frac{1}{2}}T_{eff,*}
\end{align*}

Approssimazioni successive
\begin{itemize}
    \item Correzione all'approssimazione di corpo nero: corpo grigio. Tengo conto degli effetti di riflessione (corpo nero: assorbimento perfetto).
    \item Differenza di temperatura tra regioni illuminate e non del pianeta: modello termico. Regioni polari meno illuminate cio\'e pi\'u fredde: \'e vero se l'asse di rotazione \'e ortogonale al piano dell'orbita (non vale per Urano).
    
    Per corpi con rotazione veloce o con atmosfera l'escursione termica \'e modesta.
    
    Equilibrio termico locale. Nel punto della superficie in cui il Sole \'e allo zenit: $T_{s*}=(\frac{R_*}{r_{P*}})\expy{\frac{1}{2}}T_{eff,*}$.
\end{itemize}
