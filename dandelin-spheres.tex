% Plane Sections of the Cylinder - Dandelin Spheres
% Author: Hugues Vermeiren
\documentclass{article}
\usepackage{tikz}
%%%<
\usepackage{verbatim}
\usepackage[active,tightpage]{preview}
\PreviewEnvironment{tikzpicture}
\setlength\PreviewBorder{10pt}%
%%%>
\begin{comment}
:Title: Plane Sections of the Cylinder - Dandelin Spheres
:Tags: 3D;mathematical engine;geometry;mathematics
:Author: Hugues Vermeiren
:Slug: dandelin-spheres
\end{comment}
\tikzset{
	MyPersp/.style={scale=1.8,x={(-0.8cm,-0.4cm)},y={(0.8cm,-0.4cm)},
    z={(0cm,1cm)}},
%  MyPersp/.style={scale=1.5,x={(0cm,0cm)},y={(1cm,0cm)},
%    z={(0cm,1cm)}}, % uncomment the two lines to get a lateral view
	MyPoints/.style={fill=white,draw=black,thick}
		}
\begin{document}

\begin{tikzpicture}[MyPersp,font=\large]
	% the base circle is the unit circle in plane Oxy
	\def\h{2.5}% Heigth of the ellipse center (on the axis of the cylinder)
	\def\a{35}% angle of the section plane with the horizontal
	\def\aa{35}% angle that defines position of generatrix PA--PB
	\pgfmathparse{\h/tan(\a)}
  \let\b\pgfmathresult
	\pgfmathparse{sqrt(1/cos(\a)/cos(\a)-1)}
  \let\c\pgfmathresult %Center Focus distance of the section ellipse.
	\pgfmathparse{\c/sin(\a)}
  \let\p\pgfmathresult % Position of Dandelin spheres centers
                       % on the Oz axis (\h +/- \p)
	\coordinate (A) at (2,\b,0);
	\coordinate (B) at (-2,\b,0);
	\coordinate (C) at (-2,-1.5,{(1.5+\b)*tan(\a)});
	\coordinate (D) at (2,-1.5,{(1.5+\b)*tan(\a)});
	\coordinate (E) at (2,-1.5,0);
	\coordinate (F) at (-2,-1.5,0);
	\coordinate (CLS) at (0,0,{\h-\p});
	\coordinate (CUS) at (0,0,{\h+\p});
	\coordinate (FA) at (0,{\c*cos(\a)},{-\c*sin(\a)+\h});% Focii
	\coordinate (FB) at (0,{-\c*cos(\a)},{\c*sin(\a)+\h});
	\coordinate (SA) at (0,1,{-tan(\a)+\h}); % Vertices of the
                                           % great axes of the ellipse
	\coordinate (SB) at (0,-1,{tan(\a)+\h});
	\coordinate (PA) at ({sin(\aa},{cos(\aa)},{\h+\p});
	\coordinate (PB) at ({sin(\aa},{cos(\aa)},{\h-\p});
	\coordinate (P) at ({sin(\aa)},{cos(\aa)},{-tan(\a)*cos(\aa)+\h});
     % Point on the ellipse on generatrix PA--PB

	\draw (A)--(B)--(C)--(D)--cycle;
	\draw (D)--(E)--(F)--(C);
	\draw (A)--(E) (B)--(F);
	\draw[blue,very thick] (SA)--(SB);

%	\coordinate (O) at (0,0,0);
%	\draw[->] (O)--(2.5,0,0)node[below left]{x};
%	\draw[->] (O)--(0,3,0)node[right]{y};
%	\draw[->] (O)--(0,0,6)node[left]{z};

	\foreach \t in {20,40,...,360}% generatrices
		\draw[magenta,dashed] ({cos(\t)},{sin(\t)},0)
      --({cos(\t)},{sin(\t)},{2.0*\h});
	\draw[magenta,very thick] (1,0,0) % lower circle
		\foreach \t in {5,10,...,360}
			{--({cos(\t)},{sin(\t)},0)}--cycle;
	\draw[magenta,very thick] (1,0,{2*\h}) % upper circle
		\foreach \t in {10,20,...,360}
			{--({cos(\t)},{sin(\t)},{2*\h})}--cycle;
	\fill[blue!15,draw=blue,very thick,opacity=0.5]
     (0,1,{\h-tan(\a)}) % elliptical section
		\foreach \t in {5,10,...,360}
			{--({sin(\t)},{cos(\t)},{-tan(\a)*cos(\t)+\h})}--cycle;

	\foreach \i in {-1,1}{%Spheres!
		\foreach \t in {0,15,...,165}% meridians
			{\draw[gray] ({cos(\t)},{sin(\t)},\h+\i*\p)
				\foreach \rho in {5,10,...,360}
					{--({cos(\t)*cos(\rho)},{sin(\t)*cos(\rho)},
          {sin(\rho)+\h+\i*\p})}--cycle;
			}
		\foreach \t in {-75,-60,...,75}% parallels
			{\draw[gray] ({cos(\t)},0,{sin(\t)+\h+\i*\p})
				\foreach \rho in {5,10,...,360}
					{--({cos(\t)*cos(\rho)},{cos(\t)*sin(\rho)},
          {sin(\t)+\h+\i*\p})}--cycle;
			}
					\draw[orange,very thick] (1,0,{\h+\i*\p})% Equators
		\foreach \t in {5,10,...,360}
			{--({cos(\t)},{sin(\t)},{\h+\i*\p})}--cycle;
		}
	\draw[red,very thick] (PA)--(PB);
	\draw[red,very thick] (FA)--(P)--(FB);
%	\fill[MyPoints] (CLS) circle (1pt);% center of lower sphere
%	\fill[MyPoints] (CUS) circle (1pt);% center of upper sphere
	\fill[MyPoints] (FA) circle (1pt)node[right]{$F_1$};
	\fill[MyPoints] (FB) circle (1pt)node[left]{$F_2$};
	\fill[MyPoints] (SA) circle (1pt);
	\fill[MyPoints] (SB) circle (1pt);
	\fill[MyPoints] (P) circle (1pt)node[below left]{$P$};
	\fill[MyPoints] (PA) circle (1pt)node[below right]{$P_1$};
	\fill[MyPoints] (PB) circle (1pt)node[above right]{$P_2$};
\end{tikzpicture}

\end{document}



%% document-wide tikz options and styles

\tikzset{%
  >=latex, % option for nice arrows
  inner sep=0pt,%
  outer sep=1pt,%
  mark coordinate/.style={inner sep=0pt,outer sep=0pt,minimum size=4pt,
  fill=black,circle},%
	sundot/.style={
	fill, color=yellow, circle, inner sep=1.5pt}
}


\begin{tikzpicture} % "THE GLOBE" showcase


    \def\R{0.5} % sphere radius
    \def\angEl{5} % elevation angle
    \def\angAz{105} % azimuth angle
    \def\angPhi{-40} % longitude of point P
    \def\angBeta{19} % latitude of point P

    \pgfmathsetmacro\H{\R*cos(\angEl)} % distance to north pole
    \tikzset{xyplane/.style={cm={cos(\angAz),sin(\angAz)*sin(\angEl),-sin(\angAz),
                                  cos(\angAz)*sin(\angEl),(0,-\H)}}}
    \LongitudePlane[xzplane]{\angEl}{\angAz}
    \LatitudePlane[equator]{\angEl}{0}

     \filldraw[ball color=white, fill opacity=1] (0,0) circle (\R);
    \draw (0,0) circle (\R);
    \coordinate (O) at (0,0);
    \coordinate[mark coordinate] (N) at (0,\H);
    \coordinate[mark coordinate] (S) at (0,-\H);
%		\coordinate (horizon) at (-30.0:\R);
%		\coordinate (equator) at (\H,0);


    \draw[->] (0,-\H-2) -- (0,\R+2) node[above] {}; %axis of rotation
%		\draw[->,rotate=-30.0] (0,-\H-5) -- (0,\R+5) node[above] {\bf{Zenith}}; %axis of rotation

    \path[xzplane] (\R,0) coordinate (XE);

%		\DrawLatitudeCircle[\R,color=blue]{0} % equator
%		\DrawLatitudeCircle[\R,rotate=23.5,color=red]{0}
%		\node[right,color=red] at (horizon) {\bf{Horizon}};
%		\node[above=9pt, right=-5pt, color=blue] at (equator) {\bf{Equator}};
%		\node[label={above right:\bf{Equator}}] at (equator) {};

		\path (N) node[align=left, above=1.5em, right] {\textbf{North}\\ \textbf{pole}} (N);
		\path (S) node[align=left, below=1.5em, right] {\textbf{South}\\ \textbf{pole}} (S);
%    \node[above=10pt, left=3pt] at (N) {\bf{North}};
%    \node[above=2pt, left=12pt] at (N) {\bf{pole}};

%    \node[below=5pt, right=4pt] at (S) {\bf{South}};
%    \node[below=14pt, right=4pt] at (S) {\bf{pole}};

    \def\R{3.5} % sphere radius
    \def\angEl{5} % elevation angle
    \def\angAz{105} % azimuth angle
    \def\angPhi{-40} % longitude of point P
    \def\angBeta{19} % latitude of point P

    \pgfmathsetmacro\H{\R*cos(\angEl)} % distance to north pole
    \tikzset{xyplane/.style={cm={cos(\angAz),sin(\angAz)*sin(\angEl),-sin(\angAz),
                                  cos(\angAz)*sin(\angEl),(0,-\H)}}}
    \LongitudePlane[xzplane]{\angEl}{\angAz}
    \LatitudePlane[equator]{\angEl}{0}

    \filldraw[ball color=white, fill opacity=0.15] (0,0) circle (\R);
    \draw (0,0) circle (\R);

    \coordinate (O) at (0,0);
    \coordinate[mark coordinate] (N) at (0,\H);
    \coordinate[mark coordinate] (S) at (0,-\H);
		\coordinate (Equator) at (\H,0);
		\coordinate (Ecliptic) at (23.5:\R);
    \path[xzplane] (\R,0) coordinate (XE);

		\draw[->,ultra thick] (0:\R) arc (0:23.5:\R);
		\coordinate (tilt) at (11.75:\R);
		\node[right=5pt] at (tilt) {$23.5^{\circ}$};
%		\coordinate (obslat) at (75:0.4*\R);
%		\node[above=5pt] at (obslat) {$30^{\circ}$};
		\DrawLatitudeCircleName[\R,color=blue]{0}{celeq} % equator
%		\draw 
		\DrawLatitudeCircleName[\R,rotate=23.5, color=orange]{0}{ecl} % ecliptic
		\node[right=0.5em, color=blue] at (Equator) {\textbf{Celestial equator}};
		\node[above=0.5em, right=0.5em, color=orange] at (Ecliptic) {\textbf{Ecliptic}};
		\path[name intersections={of= celeq and ecl,by={vereq}}];%, [label=above left:\textbf{Autumnal equinox}]b}}];
		\coordinate[mark coordinate] () at (vereq) {};
		\path (vereq) node[align=right, below=1.3em, right] {\textbf{Vernal equinox}\\ March 20th}
		(vereq);
		\coordinate[mark coordinate] (sumsol) at (23.5:\R);
		\path (sumsol) node[align=right, above=1.3em, left=1em] {\textbf{Summer solstice}\\ June
		21st} (sumsol);
		\coordinate[mark coordinate] (winsol) at (23.5+180:\R);
		\path (winsol) node[align=right, below=1.5em, right=0.5em] {\textbf{Winter solstice}\\
		December 21st} (winsol);
		\coordinate[mark coordinate] (auteq) at (100.5:0.09*\R);
		\path (auteq) node[align=right, above=1.3em, left=0.5em] {\textbf{Autumnal equinox}\\
		September 23rd} (auteq);
		\coordinate[sundot] (sun) at (15:0.5*\R);
		\draw[->,very thick] (sun) -- (18.4:0.7*\R);
		\node[above=1.0em,right=0.1em] at (vereq) {$\aries$};
		\node[below=1.0em] at (sun) {\textbf{The sun}};
%		\draw[->] (auteq) -- (120:0.3*\R);
%		\node[mark coordinate,align=center,below right] at (a) {Vernal \\ equinox};
%		\node[mark coordinate,label=above left:\textbf{Summer solstice}]
%		\node[mark coordinate] at (a) {};
%		\coordinate[mark coordinate] (a) at (intersection-2);
%		\node[label={above right:\bf{Celestial horizon}},color=red] at (Horizon) {};
%		\node[label={right:\bf{Celestial equator}},color=red] at (Equator) {};

%		\DrawLongitudeCircle[\R]{\angAz+15}{} % xzplane

    \node[above=7pt, left=5pt] at (N) {\bf{North celestial pole}};
    \node[below=8pt, right=5pt] at (S) {\bf{South celestial pole}};
    
    \def\R{3.5} % sphere radius
    \def\angEl{5} % elevation angle
    \def\angAz{105} % azimuth angle
    \def\angPhi{-40} % longitude of point P
    \def\angBeta{19} % latitude of point P
\DrawLatitudeCircleName[\R,rotate=23.5+13, color=green]{0}{orb} % piano dell'orbita


\end{tikzpicture}
