\subsection{Overview terrestrial planets}

%solarsystemmain-orbit: dps e i P% spin
%solarsystemmain-pc: mass 	  rhomean Rp   gs  j2
%solarsystemmain-th: A 	 Teff TeffA Tss  H	  atms
%display columns/1/.style={column name={$r_{p\odot}(\SI{e8}{\kilo\meter})$},clear infinite},
%display columns/2/.style={column name={$R_p\SI{e3}{\kilo\meter}$},clear infinite},
%display columns/3/.style={column name={albedo},clear infinite},
%display columns/4/.style={column name={$T_{eff,p}$},clear infinite},
%display columns/5/.style={column name={$T_{eff,p}'$},clear infinite},
%display columns/6/.style={column name={$T_{ss}$},clear infinite},
%display columns/7/.style={column name={$m_V^{opp}$},clear infinite},
%display columns/8/.style={column name={H},clear infinite},


\begin{frame}{Caratteristiche principali}
\begin{table}[!ht]
\pgfplotstabletypeset[skip rows between index={4}{8},
every head row/.style={
 %before row={},
 %every last row/.style={after row=\bottomrule},
 after row={\midrule}
},
every 2 row/.style={after row=\midrule},
every last row/.style={after row=\bottomrule},
every first column/.style={column type/.add={|}{}},
every last column/.style={column type/.add={}{|}},
%columns/0/.style = {column type/.add={|}{}},
%columns/5/.style = {column type/.add={|}{}},
%columns/0/.style={string type},
display columns/1/.style={column name={$r_{p\odot}(\SI{e8}{\kilo\meter})$},clear infinite},
display columns/2/.style={column name={e},clear infinite},
display columns/3/.style={column name={i},clear infinite},
display columns/4/.style={column name={$\tau_{sid}$},clear infinite},
display columns/5/.style={column name={obliquity},clear infinite},
display columns/planets/.style={column name={pianeta},string type},
create on use/planets/.style={create col/set list={Mercury, Venus, Earth, Mars}},
columns/planets/.style={string type},
columns={planets, dps, e, i, Ps, obl},
/pgf/number format/precision=3
     ]{solarsystemmain-orbit.txt} %%%
%\captionof{table}{Caratteristiche pianeti terrestri.}\label{tab:terrestrial planets}
\begin{itemize}
\item Accretion of planetesimal
\item Ratation depends upon fine details (i, e,\ldots)
\end{itemize}
\end{table}
\end{frame}

\begin{wordonframe}{Pianeti terrestri: overview}
\begin{itemize}\item Hill interior/exterior orbits are symmetric: positive/negative angular momentum is equally likely
\item Formation from planetesimal accretion; slow accretion deprived the regions from material to form satellites.

\end{itemize}
\end{wordonframe}

\subsection{Mercury}

%\begin{frame}{Mercurio:}
%\end{frame}
%\begin{wordonframe}{struttura}
%\end{wordonframe}

\subsection{Venus}

%\begin{frame}{Venus: Spin/runaway greenhouse}
%\end{frame}
%\begin{wordonframe}{struttura}
%\end{wordonframe}

\subsection{Sistema Terra-Luna}
\begin{frame}{Sistema Terra/Luna}
\tolbf
Formation scenarios:
\end{frame}

\begin{wordonframe}{Terra/Luna}

\end{wordonframe}

\subsection{Marte}

%\begin{frame}{Mars: satellites}
%Phobos, Deimos: tidal evolution inward outward
%\end{frame}
%\begin{wordonframe}{struttura}
%\end{wordonframe} 